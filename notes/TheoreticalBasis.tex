\documentclass[a4paper, 12pt]{article}
\usepackage[a4paper, top=3cm, bottom=3cm, left=3cm, right=3cm, marginparwidth=1.75cm]{geometry}
\usepackage[utf8x]{inputenc}

% fyrir myndir
\usepackage{float}
\usepackage{graphicx}
\usepackage{caption}
\usepackage{wrapfig}
\usepackage{nicefrac}
\usepackage{gensymb}
\captionsetup{font=small, labelfont={bf,sf}}

\usepackage[colorlinks=False]{hyperref}

\usepackage{amsmath, amsfonts, amssymb} 
\usepackage{mathrsfs}
\usepackage{color}
\usepackage{enumitem}
\usepackage[version=4]{mhchem}
\usepackage{icomma}
\usepackage{listings}
\lstset{
    language=Python,
    basicstyle=\ttfamily,
    breaklines=true, 
    columns=flexible,
    numbers=left,
    numberstyle=\ttfamily,
    showstringspaces=false
}

\title{
    The SimEVO project
} 
\author{Kári Hlynsson}
\date{\small \lstinline{https://github.com/lvthnn/SimEVO}}

\begin{document}
    \maketitle

    \begin{abstract}
        \noindent We discuss the theoretical basis of the SimEVO simulation project
        and how it can be used to obtain data for use in biological research. Mathematical
        notation is covered and some useful methods of analysis are proposed. We discuss some
        ideas on how to obtain and process data in order to deduce conclusions regarding forces of
        selection and evolutionary processes. For more information on installation or usage, see the
        \lstinline{README.md} file located in the GitHub repository.
    \end{abstract}

    \tableofcontents

    \section{Introduction}
    Write introduction please.

    \section{Theoretical basis}
    When Charles Darwin published
    
\end{document}